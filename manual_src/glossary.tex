%\documentclass{article}

%\begin{document}
\chapter{Basic Definitions}
\label{cha:glossary}
\begin{description}
\item[Arrival Rate ($\lambda$).] Rate at which customers of a given open
class arrive in the system.  %Each class has its own workload intensity. If a class is closed the constant \emph{number of customers} $N$ that are present in the system must be provided, if a class is open the \emph{arrival rate} $\lambda$ of customers to the system is requested.
\item[Batch workload.] Closed workload class composed of customers that do not require interaction with delay stations during their execution (i.e., the users think time is zero).
%Its intensity is specified by a parameter indicating the average number of executing jobs. %A batch workload has a fixed population. Jobs that have completed service can be thought of as leaving and being replaced instantaneously from a backlog of waiting jobs.
\item[Bottleneck Station(s).] The station(s) with the highest
utilization in the system.
%\item[Bottleneck.] The bottleneck resource is the first to saturate. In a model with load independent centers, the bottleneck coincides with the resource with the largest service demand ($D_i$).
\item[Class of customers.] A group of customers with service
demands on the different stations statistically equal. Open
classes are specified by an \emph{arrival rate} (job/s), closed
classes are composed by a fixed number of jobs specified as a
\emph{population} (job). \item[Closed Class.] Workload class in
which customers that have completed service leave the system and
are instantaneously replaced by a new customer. A closed model has
thus a fixed population of requests. \item[Customer] or
\textbf{Job} or \textbf{Transaction} or \textbf{Request}. It is
the element that will require service from network stations. For
example, it can be an {http request}, a {database query},a {ftp
download command}, or an I/O request. \item[Delay station.]
Customers at a delay station are each (logically) allocated their
own server, so there is no competition for service (no queue). The
time spent by a customer at a delay station is exactly its service
demand. In delay stations, for consistency with Little's law, the
\emph{utilization} is computed as the \emph{average number of
customers} in the station, and thus it may be \emph{greater than
1}

\item[Dominated Stations.] A station P is dominated if exist al
least one station Q whose service demands for each class of
customer are highest or equal (but at least one highest) than
those of P. \item[Interactive workload (terminal workload).] Its
intensity is specified by two parameters: the number of active
terminals (customers), and the average length of time that
customers use terminals ("think time") between interactions.
\item[Load Dependent Resource (station).] A load dependent service
station can be thought of as a service station whose service rate
(the reciprocal of its service time) is a function of the customer
population in its queue. \item[Masked-off Stations.] a station
that is not a Potential Bottleneck or a Dominated station. A
Masked-off station may exhibit the largest queue-length and hence
the highest response time. \item[Mean Value Analysis (MVA).]
Iterative technique used to evaluate closed queuing networks.
\item[Multiple class models.] Models with several customer
classes, each of which has its own workload intensity and its own
service demand at each center. Within each class, the customers
are indistinguishable. \item[Number of customers at a resource $i$
($Q_i$).] The average queue length at station $i$ include all
customers at that center, whether waiting in queue or receiving
service. \item[Number of customers in the system.] It is the
aggregate measure of \emph{Queue Length} over all stations.
\item[Number of visits.] The average number of visits that a
customer makes to each resource during a complete execution.
\item[Open Class]  or {\bf Transaction Workload}. Workload class
in which there is an external source with an arbitrary number of
customers that are sent to the system according to a given
distribution. The number of customers in the model varies over
time and can be arbitrarily large depending on the congestion
level of the resources and on the arrival process. \item[Global
Population ($N$).] Constant number of requests belonging to a
closed class. \item[Population Mix ($\beta_i$).] The ratio between
$N_i$, the \emph{number of customers} of closed class $i$, and the
total number of customers in the system: $\beta_i = N_i / \sum_k
N_k$ \item[Potential Bottleneck Station.] a station that can
become bottleneck for some feasible combination of population
\item[Queue length ($Q_k$), also referred to as Customer number.]
average number of customers at station $k$, either waiting in
queue \emph{and} receiving service. \item[Queueing network model.]
A network of queues is a collection of service resources, which
represent system resources, and customers, which represent users
or transactions. \item[Queueing station.] A resource consisting of
two components: queue and server. Customers at a queueing resource
compete for the use of the server. The time spent by a customer at
a queueing resource has \emph{two} components: time spent waiting
in queue and time spent receiving service. \item[Reference
Station.] User-specified station where a customer flow through
when it has completed its execution, i.e., it has performed an
entire cycle of service in the system. The system throughput,
system response time, and the visits $V_k$ are computed with
respect to the visits at the reference station (\emph{usually}
assumed as 1). \item[Residence Time ($W_k$).] Average time that a
customer spent at station $k$ during its complete execution. It
includes time spent queueing and time spent receiving service. It
does not correspond to \emph{Response Time} $R_k$ of a station
since $W_k = R_k * V_k$. \item[Response Time ($R_k$).] Average
time required by a customer to flow through a station $k$
(includes queue time and service time).
\item[Saturated Resource.]
When the arrival rate at a resource is greater than or equal to
the maximum service rate, the resource is said to be saturated (i.
e., its utilization is equal to 1). \item[Saturation Sector.] a
interval of Population Mix in witch one, two ore more stations are
both bottleneck. \item[Server Utilization ($U$).] The utilization
of a queuing center is the proportion of time the resource is busy
or, equivalently, the average number of customers in service there
(definition valid also for delay centers). \item[Service Demand
($D_k$).] average service requirement of a customer, that is the
total amount of service required  by a complete execution at
resource $k$. In the model it is necessary to provide separate
service demand for each pair service center-class. It is given by
$D_k = V_k * S_k$. \item[Service Time ($S_k$).] Average service
requirement of a request of a given class per visit at resource
$k$. \item[Station] or \textbf{Service Center} or
\textbf{Resource}. It represents a unit of the network. Customers
arrive at the station and then, if necessary, wait in queue,
receive service from server, and eventually depart from the
station. \item[System Response Time ($R$).] Correspond to the
intuitive notion of response time perceived by users, that is, the
time interval between the instant of the submission of a request
to a system and the instant the corresponding reply arrives
completely at the user. It is the aggregate measure of
\emph{Residence Times}: $R= \sum_k W_k$. \item[System Throughput
($X$).] Rate at which customers perform an entire interaction with
the system. It is the throughput observed at a user-defined
reference station. \item[Throughput ($X_k$).] Rate at which
customers are executed by station $k$ accounting also periods
where the server is idle. That is $X_k$ is the mean number of
completions in a time unit.
%\item[] Its population varies overtime. Customers that have completed service leave the model. Its intensity is specified by a parameter indicating the rate at which requests (customers) arrive at the model.
\item[Utilization ($U_k$).] Proportion of time in which the station $k$ is busy or, in the case of a delay center. For single server stations it may be interpreted as the average  number of  customers in the station (see Little Law \cite{Little}).
\item[Visit (number of -) ($V_k$).] Average number of visits that a customer makes at station $k$ during a complete execution; it is computed as the ratio of the number of completions at resource $k$ to the number of system completions as observed at a user-defined reference station. If resource $k$ is a delay center representing a client station, it is a convention assign a unitary value to number of visits to this station.
\end{description}
%\end{document}
